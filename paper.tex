\documentclass[12pt,letterpaper]{article}
\usepackage{ifpdf}
\usepackage{mla}
\begin{document}
\begin{mla}{Joseph}{Pollard}{Jason}{Classics 2220}{\today}{}
For six days and [seven (?)] nights the wind blew, flood and tempest overwhelmed the land; when the seventh day arrived, the tempest, flood and onslaught, which had struggled like a woman in labor, blew themselves out (?). The sea became calm, the \emph{imhullu}-wind grew quiet, the flood held back. I looked at the weather; silence reigned, for all mankind had returned to clay. The flood-plain was flat as a roof. I opened a porthole and light fell on my cheeks. I bent down, then sat. I wept. My tears ran down my cheeks. I looked for banks, for limits to the sea. Areas of land were emerging everywhere (?). Pg 83

	The excerpt above is just one of the many examples of ``global'' flooding that is depicted in ancient religious texts. This is from The Epic of Gilgamesh, but it has several common elements with many texts from different cultures. Usually the flood is depicted as a weapon or evil that kills all of mankind. Typically there is one god that prevents the flood from killing everyone or helps a few prepare for the floods arrival. Lastly, the similarities between this particular text and the Hebrew/Christian story of Noah demonstrates the incredible amount of influence that the cultures and religions had upon one another.  
\end{mla}
\end{document}
